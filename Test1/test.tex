\documentclass{article}
\usepackage[table]{xcolor}
\usepackage{colortbl}
\usepackage{booktabs}
\usepackage{siunitx}
\usepackage{graphicx}
\usepackage{amssymb}

\sisetup{
    table-format=3.2,
    round-mode=places,
    round-precision=2
}


\begin{document}

\begin{table}[ht!]
    \centering
    \caption{A1a: corrected Hall voltage, magnet current, and field}
    \begin{tabular}{| S[table-format=2.1] | S[table-format=1.3] | S[table-format=2.2] |}
    \toprule
    {$U_H/mV$} & {$I_B/A$} & {$B/mT$} \\
    \midrule
    {$9.0 \pm 0.1$} & {$0.498 \pm 0.009$} & {$72 \pm 6$} \\
    {$10.21 \pm 0.07$} & {$0.58 \pm 0.01$} & {$84 \pm 6$} \\
    {$11.36 \pm 0.08$} & {$0.66 \pm 0.02$} & {$94 \pm 6$} \\
    {$12.48 \pm 0.08$} & {$0.72 \pm 0.02$} & {$103 \pm 6$} \\
    {$13.71 \pm 0.08$} & {$0.8 \pm 0.6$} & {$110 \pm 70$} \\
    {$14.95 \pm 0.09$} & {$0.87 \pm 0.02$} & {$124 \pm 6$} \\
    {$16.34 \pm 0.09$} & {$0.95 \pm 0.02$} & {$135 \pm 6$} \\
    {$17.55 \pm 0.09$} & {$1.03 \pm 0.02$} & {$144 \pm 6$} \\
    {$18.9 \pm 0.1$} & {$1.10 \pm 0.02$} & {$155 \pm 6$} \\
    {$20.3 \pm 0.1$} & {$1.18 \pm 0.02$} & {$166 \pm 6$} \\
    {$21.4 \pm 0.2$} & {$1.25 \pm 0.02$} & {$175 \pm 6$} \\
    {$22.7 \pm 0.2$} & {$1.33 \pm 0.02$} & {$185 \pm 6$} \\
    {$24.1 \pm 0.2$} & {$1.40 \pm 0.02$} & {$196 \pm 6$} \\
    {$25.3 \pm 0.2$} & {$1.48 \pm 0.02$} & {$206 \pm 6$} \\
    {$26.6 \pm 0.2$} & {$1.55 \pm 0.07$} & {$220 \pm 10$} \\
    \bottomrule
    \end{tabular}
    \label{tab:A1a}
\end{table}

\begin{table}[ht!]
    \centering
    \caption{A1b: source current vs corrected Hall voltage}
    \begin{tabular}{| S[table-format=2.1] | S[table-format=2.1] |}
    \toprule
    {$U_H/mV$} & {$I_S/mA$} \\
    \midrule
    {$26.5 \pm 0.2$} & {$20.0 \pm 0.5$} \\
    {$25.7 \pm 0.2$} & {$19.0 \pm 0.5$} \\
    {$23.8 \pm 0.1$} & {$18.0 \pm 0.5$} \\
    {$22.8 \pm 0.2$} & {$17.0 \pm 0.5$} \\
    {$21.12 \pm 0.09$} & {$16.0 \pm 0.5$} \\
    {$18.4 \pm 0.2$} & {$14.0 \pm 0.5$} \\
    {$16.98 \pm 0.07$} & {$13.0 \pm 0.5$} \\
    {$14.08 \pm 0.06$} & {$11.0 \pm 0.5$} \\
    {$12.42 \pm 0.05$} & {$10.0 \pm 0.5$} \\
    {$10.91 \pm 0.05$} & {$9.0 \pm 0.5$} \\
    {$9.02 \pm 0.04$} & {$8.0 \pm 0.5$} \\
    {$8.76 \pm 0.08$} & {$7.0 \pm 0.5$} \\
    {$6.94 \pm 0.03$} & {$6.0 \pm 0.5$} \\
    {$5.67 \pm 0.03$} & {$5.0 \pm 0.5$} \\
    {$2.76 \pm 0.06$} & {$3.0 \pm 0.5$} \\
    {$0.98 \pm 0.02$} & {$2.0 \pm 0.5$} \\
    \bottomrule
    \end{tabular}
    \label{tab:A1b}
\end{table}

\begin{table}[ht!]
    \centering
    \caption{Diameters of the pinholes and the calculated minimum diameter for each.}
    \begin{tabular}{| l | S[table-format=2.1] | S[table-format=1.3] |}
    \toprule
    {names} & {$diameter\ of\ pinholes\ (mm)$} & {$d_{min}\ (mm)$} \\
    \midrule
    {$A_1$} & {$0.2$} & {$0.22$} \\
    {$A_2$} & {$1.0$} & {$0.044$} \\
    {$B_1$} & {$0.3$} & {$0.15$} \\
    {$B_2$} & {$0.6$} & {$0.073$} \\
    {$B_3$} & {$0.4$} & {$0.11$} \\
    \bottomrule
    \end{tabular}
    \label{tab:ex4}
\end{table}


\begin{table}[ht!]
    \centering
    \caption{First, $t$ is fixed at $15\,\mathrm{cm}$ while \ensuremath{x^\prime} is varied to find $x_{\mathrm{best}}$; then \ensuremath{x^\prime} is fixed and $t$ is varied. Magnification is shown from both theory and experiment.}
    \resizebox{\textwidth}{!}{
    \begin{tabular}{| S[table-format=2.1] | S[table-format=2.1] | S[table-format=2.1] | S[table-format=2.1] | S[table-format=2.1] | S[table-format=2.1] | S[table-format=2.1] | S[table-format=2.1] | S[table-format=2.1] |}
    \toprule
    {$t$\ (cm)} & {$\tilde{x}$\ (cm)} & {$x$\ (cm)} & {$G$} & {$B$} & {$\beta_{ob}$} & {$t/f$} & {$\Gamma_{th}$} & {$\Gamma_{ex}$} \\
    \midrule
    {$15.0 \pm 0.1$} & {$16.9 \pm 0.1$} & {$12.0 \pm 0.2$} & {$3.0 \pm 0.5$} & {$10.0$} & {$3.3 \pm 0.6$} & {$3.75 \pm 0.03$} & {$27.2 \pm 0.2$} & {$24 \pm 5$} \\
    {} & {$18.3 \pm 0.1$} & {$13.4 \pm 0.2$} & {$2.5 \pm 0.5$} & {$10.0$} & {$4.0 \pm 0.8$} & {$3.75 \pm 0.03$} & {$27.2 \pm 0.2$} & {$29 \pm 6$} \\
    {} & {$21.0 \pm 0.1$} & {$16.1 \pm 0.2$} & {$1.5 \pm 0.5$} & {$10.0$} & {$7 \pm 3$} & {$3.75 \pm 0.03$} & {$27.2 \pm 0.2$} & {$50 \pm 20$} \\
    \midrule
    {$20.0 \pm 0.1$} & {$18.3 \pm 0.1$} & {$13.4 \pm 0.2$} & {$1.5 \pm 0.5$} & {$10.0$} & {$7 \pm 3$} & {$5.00 \pm 0.03$} & {$36.2 \pm 0.2$} & {$50 \pm 20$} \\
    {$30.0 \pm 0.1$} & {$18.3 \pm 0.1$} & {$13.4 \pm 0.2$} & {$5.0 \pm 0.5$} & {$50.0$} & {$10 \pm 1$} & {$7.50 \pm 0.03$} & {$54.4 \pm 0.2$} & {$72 \pm 8$} \\
    \bottomrule
    \end{tabular}
    }
    \label{tab:ex2}
\end{table}


\begin{table}[ht!]
    \centering
    \caption{Carrier density $n$ and mobility $\mu$ for selected materials.}
    \begin{tabular}{| >{\columncolor{black!20}}l | S[table-format=2.2] | S[table-format=4.0] |}
    \toprule
    \multicolumn{1}{|c|}{ID} & \multicolumn{2}{|c|}{Transport} \\
    \midrule
    {Material} & {$n$ ($\mathrm{cm^{-3}}$)} & {$\mu$ ($\mathrm{cm^2/Vs}$)} \\
    \midrule
    {Al} & {$(27.1 \pm 1.0)e18$} & {$1500 \pm 50$} \\
    {Cu} & {$(84.9 \pm 5.0)e21$} &  \\
    {Si} & {$(20.0 \pm 1.0)e9$} &  \\
    {Ge} & {$(23.0 \pm 1.0)e12$} &  \\
    \bottomrule
    \end{tabular}
    \label{tab:transport}
\end{table}

\begin{table}[ht!]
    \centering
    \caption{Two runs with per-block uncertainties (block errors override header errors).}     
    \begin{tabular}{| S[table-format=2.0] | S[table-format=2.2] | >{\columncolor{black!20}}l |}
    \toprule
    \multicolumn{2}{|c|}{Run A} & \multicolumn{1}{|c|}{Config} \\
    \midrule
    {$T$ (K)} & {$R$ (\si{\ohm})} & {Bias} \\
    \midrule
    {$300.0 \pm 0.2$} & {$10.010 \pm 0.050$} & {} \\
    {$320.0 \pm 0.2$} & {$9.560 \pm 0.050$} & {} \\
    {$340.0 \pm 0.2$} & {$9.120 \pm 0.040$} & {} \\
    \midrule
    {$360.0 \pm 0.2$} & {$8.790 \pm 0.020$} & {$1$} \\
    {$380.0 \pm 0.2$} & {$8.510 \pm 0.020$} & {} \\
    \bottomrule
    \end{tabular}
    \label{tab:runs}
\end{table}


\pagebreak



\[
\textbf{Claim.}\quad \sum_{k=0}^{N-1} \mathbf v_k= \mathbf 0,\qquad
\mathbf v_k = R\big(\cos\phi_k\,\hat{\mathbf x}+\sin\phi_k\,\hat{\mathbf y}\big),\quad N\ge2.
\]


\textbf{Proof 1 (roots of unity).}
Identify \(\mathbf v_k\) with the complex number \(R e^{i\phi_k}=R e^{i\phi_0}\omega^k\),
where \(\omega=e^{2\pi i/N}\).
Then
\[
\sum_{k=0}^{N-1}\mathbf v_k
= R e^{i\phi_0}\sum_{k=0}^{N-1}\omega^k
= R e^{i\phi_0}\,\frac{1-\omega^{N}}{1-\omega}=0,
\]
since \(\omega^N=1\) and \(\omega\ne1\) for \(N\ge2\).
Thus the vector sum is \(\mathbf 0\).

\textbf{Proof 2 (rotation symmetry).}
Let \(S=\sum_{k=0}^{N-1}\mathbf v_k\).
Rotate every vector by \(2\pi/N\) (which permutes the set), so the sum is unchanged:
\(S\) maps to \(e^{i(2\pi/N)}S\) but must still equal \(S\).
Hence \((1-e^{i2\pi/N})S=\mathbf 0\).
For \(N\ge2\), \(e^{i2\pi/N}\ne1\), so \(S=\mathbf 0\).

\(\square\)

\end{document}