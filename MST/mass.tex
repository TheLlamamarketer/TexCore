\documentclass[11pt]{article}
\usepackage[margin=1in]{geometry}
\usepackage[utf8]{inputenc}
\usepackage{amsmath,amssymb}
\usepackage{url}
\usepackage{graphicx}
\usepackage{enumitem}
\usepackage[colorlinks=true]{hyperref}
\usepackage{bookmark}
\usepackage{parskip}
\usepackage[backend=biber,style=authoryear]{biblatex}
\addbibresource{references.bib}

\title{Massenspektrometrie}
\author{}
\begin{document}
\maketitle



\section*{Genral Notes}

\begin{itemize}[leftmargin=*]
    \item Masse zu Ladung verhältnis wird gemessen und ein Massenspektrum erstellt
    \item Die zu analysierende Substanz wird ionisiert
    \item Verschiedene Ionisierungsmethoden (ESI, MALDI, EI, CI)
    \item Diese Ionen werden in einem Massenspektrometer nach ihrem m/z getrennt
    \item Nur die positiven Ionen werden zum Detektor geleitet, also die Kationen
    \item Die Anzahl an zu analysierenden Molekülen wird von den Analysator begrenzt, wegen verschiedenen Ionisierungsmethoden (\url{https://userpage.fu-berlin.de/~springer/guide_eng.html})
\end{itemize}


\section*{Quick outline}
\begin{itemize}[leftmargin=*]
    \item Motivation: why MS (sensitivity, specificity, molecular weight, structural info)
    \item Sample preparation (matrix effects, ion suppression, cleanup)
    \item Ionization methods (ESI, MALDI, EI, CI)
    \item Mass analyzers (TOF, Quadrupole, Orbitrap, FT-ICR)
    \item Tandem MS and fragmentation (MS/MS strategies)
    \item Data interpretation and reporting (m/z, isotopic patterns, mass accuracy)
    \item Common pitfalls and troubleshooting
    \item Take-home messages
\end{itemize}

\section*{Key concepts (short)}
\begin{itemize}[leftmargin=*]
    \item m/z: mass-to-charge ratio; observed peaks indexed by charge state.
    \item Mass accuracy: difference between measured and theoretical mass (ppm).
    \item Resolving power: \(R=\dfrac{m}{\Delta m}\) (often defined at FWHM).
    \item Isotopic distributions reveal elemental composition clues.
    \item Tandem MS provides structural information via controlled fragmentation.
    \item Calibration and internal standards improve quantitation and accuracy.
\end{itemize}

\section*{Ionization methods (one-line reminders)}
\begin{itemize}[leftmargin=*]
    \item ESI: soft ionization, good for polar/large biomolecules, generates multiply charged ions.
    \item MALDI: pulsed laser, solid matrix, often singly charged, good for peptides/proteins.
    \item EI/CI: gas-phase small molecules, good library matching (GC-MS).
\end{itemize}

\section*{Mass analyzers (strengths / when to mention)}
\begin{itemize}[leftmargin=*]
    \item TOF: high mass range, fast acquisition.
    \item Quadrupole: robust, good for targeted quantitation (SRM/MRM).
    \item Orbitrap / FT-ICR: very high resolving power and accuracy.
    \item Ion traps: MSn capability but limited mass range/resolution.
\end{itemize}

\section*{Tandem MS notes}
\begin{itemize}[leftmargin=*]
    \item Collision-induced dissociation (CID) basics: b/y ions for peptides.
    \item Use MS/MS to confirm identity and reduce false positives.
    \item Design experiments: targeted (SRM) vs discovery (DDA/DIA).
\end{itemize}

\section*{Common pitfalls / troubleshooting}
\begin{itemize}[leftmargin=*]
    \item Contamination: plasticizers, salts, detergents -- degrade spectra.
    \item Ion suppression: co-eluting species reduce sensitivity.
    \item Incorrect charge assignment: check isotopic spacing.
    \item Over-interpretation of low S/N peaks.
\end{itemize}

\section*{Presentation tips (for slides)}
\begin{itemize}[leftmargin=*]
    \item Show one clear spectrum and annotate peaks (m/z, charge, assignment).
    \item Use schematic diagrams for ionization and analyzer operation.
    \item Keep equations minimal: show m/z and resolving power only.
    \item Include a short example workflow (sample → prep → run → analyze).
\end{itemize}

\section*{Take-home messages}
\begin{itemize}[leftmargin=*]
    \item MS answers both identity and quantity questions when used correctly.
    \item Instrument choice depends on the question: high resolution vs robustness.
    \item Good sample prep and controls often matter more than marginal instrument upgrades.
\end{itemize}


\end{document}